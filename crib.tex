\documentclass[a4paper]{article}
\usepackage{amsmath, amsthm, amssymb, multicol}
\usepackage[margin=0.5in]{geometry}

\pagenumbering{gobble}

\begin{document}

\begin{multicols}{2}

	\section{Informal Proofs}

	\section{Proofs by Induction}

	\section{Number Theory}

	\section{Set Theory}

	\section{Sequences and Series}

	\section{Combinatronics}
	\subsection{Permutations and Combinations}
	\subsection{Pigeonhile Principle}

	\section{Discrete Probability}
	\subsection{Laplace's Theory or Probability}
	\subsection{Probability Distributuions}
	\subsection{Conditional Probability}
	\subsection{Bayes' Theorem}
	\subsection{Independence}
	\subsection{Random Variables and Expectation}

	\section{Models of Computation}
	\subsection{Regular Languages}
	\subsection{Finite Automata and Regular Expressions}
	\subsection{Contex-Free Languages and Grammarss}
	\subsection{Turning Machines}

	\section{Limitations of Computational Models}
	\subsection{Countable vs. Uncountable Sets}
	\subsection{Languages}

	\section{Decision Problems}

	\section{P and NP}
	$$P \subset NP$$
	\begin{itemize}
		\item Every	language	that	can	be	decided	by	a	deterministic	Turing
		      Machine	in polynomial	time	can	be	decided	by	a	nondeterministic
		      Turing Machine	in polynomial	time.
		\item Every	deterministic	Turing Machine	is	a	nondeterministic	TM.
		\item Is $P = NP$? We don't know.
		\item Does the satisfiability problem have a polynomial time deterministic
		      algorithm? We don't know.
		\item Cook's Theorem: The satisfiability problem is $NP$-complete.
	\end{itemize}

\end{multicols}

\end{document}
