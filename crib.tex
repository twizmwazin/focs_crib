\documentclass[a4paper]{article}
\usepackage{amsmath, amsthm, amssymb, multicol}
\usepackage[margin=0.5in]{geometry}

\pagenumbering{gobble}

\begin{document}

\begin{multicols}{2}

	\section{Informal Proofs}
	Can you speak English? Great! Informal proofs involve
	proving stuff with a combination of math and English.
	Now go out there and spout bullshit.

	\section{Proofs by Induction}

	\section{Number Theory}

	\section{Set Theory}

	\section{Sequences and Series}

	\section{Combinatronics}
	\subsection{Permutations and Combinations}
	\subsection{Pigeonhile Principle}

	\section{Discrete Probability}
	\subsection{Laplace's Theory or Probability}
	\begin{itemize}
		\item Given $S$ is a finite set of equally likely outcomes, and $E$ is an event of $S: p(E) = |E|/|S|$
		\item The probability of not $p(E) = 1 - p(E)$. This is the complement of $p(E)$.
		\item Given $E_1$ and $E_2$ are events in S: $p(E_1 U E_2) = p(E_1) + p(E_2) - p(E_1 \cap E_2)$
	\subsection{Probability Distributions}
	\begin{itemize}
		\item Let $S$ be a finite set of outcomes. We can assign a $p(s)$ to each $s$ in $S$ on 2 conditions:
		\begin{enumerate}
			\item $0 \leq p(s) \leq 1$ for each $s$
			\item $\sum_{s \in S}p(s) = 1$
		\end{enumerate}
		\item $p(s)$ is a function called the \textit{probability distribution}
	\end{itemize}
	\subsection{Conditional Probability}
	\subsection{Bayes' Theorem}
	\subsection{Independence}
	\subsection{Random Variables and Expectation}

	\section{Models of Computation}
	\subsection{Regular Languages}
	\subsection{Finite Automata and Regular Expressions}
	\subsection{Contex-Free Languages and Grammarss}
	\subsection{Turning Machines}

	\section{Limitations of Computational Models}
	\subsection{Countable vs. Uncountable Sets}
	\subsection{Languages}

	\section{Decision Problems}

	\section{P and NP}
	$$P \subset NP$$
	\begin{itemize}
		\item Every	language	that	can	be	decided	by	a	deterministic	Turing
		      Machine	in polynomial	time	can	be	decided	by	a	nondeterministic
		      Turing Machine	in polynomial	time.
		\item Every	deterministic	Turing Machine	is	a	nondeterministic	TM.
		\item Is $P = NP$? We don't know.
		\item Does the satisfiability problem have a polynomial time deterministic
		      algorithm? We don't know.
		\item Cook's Theorem: The satisfiability problem is $NP$-complete.
	\end{itemize}

\end{multicols}

\end{document}
