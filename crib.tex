\documentclass[a4paper]{article}
\usepackage{amsmath, amsthm, amssymb, multicol}
\usepackage[margin=0.5in]{geometry}

\pagenumbering{gobble}

\begin{document}

\begin{multicols}{2}

	\section{Informal Proofs}
	Can you speak English? Great! Informal proofs involve
	proving stuff with a combination of math and English.
	Now go out there and spout bullshit.

	\section{Proofs by Induction}

	\section{Number Theory}

	\section{Set Theory}

	\section{Sequences and Series}

	\section{Combinatronics}
	\subsection{Permutations and Combinations}
	\subsection{Pigeonhile Principle}

	\section{Discrete Probability}
	\subsection{Laplace's Theory or Probability}
	\subsection{Probability Distributuions}
	\subsection{Conditional Probability}
	\subsection{Bayes' Theorem}
	\subsection{Independence}
	\subsection{Random Variables and Expectation}

	\section{Models of Computation}
	\subsection{Regular Languages}
	\subsection{Finite Automata and Regular Expressions}
	\subsection{Contex-Free Languages and Grammarss}
	\subsection{Turning Machines}

	\section{Limitations of Computational Models}
	\subsection{Countable vs. Uncountable Sets}
	\subsection{Languages}

	\section{Decision Problems}
	\begin{itemize}
		\item \textbf{Decision problems} are problems with a YES or NO answer.
		\item A language is \textbf{Turing-decidable} if some Turing Machine decides
					it.
		\item The Turing Machine always halts.
		\item Not all languages are recognizeable by a turing machine.
		\item Theorem: The membersip problem is undecidable. There is no Turing
					Machine that can decide whether $M$ accepts $w$ for all $M$ and $w$.
		\item Theorem: The halting probelm is undecidable. There is no Turing
					machine that can determine whether $M$ halts on input for $w$ for all
					$M$ and $w$ (Proved in Turing's 1936 paper).
	\end{itemize}

	\section{P and NP}
	$$P \subset NP$$
	\begin{itemize}
		\item Every language that can be decided by a deterministic Turing
					Machine in polynomial time can be decided by a nondeterministic
					Turing Machine in polynomial time.
		\item Every deterministic Turing Machine is a nondeterministic Turing
					Machine.
		\item Is $P = NP$? We don't know.
		\item Does the satisfiability problem have a polynomial time deterministic
					algorithm? We don't know.
		\item Cook's Theorem: The satisfiability problem is $NP$-complete.
	\end{itemize}

\end{multicols}

\end{document}
